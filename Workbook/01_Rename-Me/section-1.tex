% section-1
% Copyright © 2018-2022 NextStep IT Training, a division of Smallrock Internet Services, Inc. All rights reserved.
%
% [Replace me: this is a chapter section, describe what is going on here.]
%

\providecommand{\main}{..}
\documentclass[../workbook]{subfiles}
\graphicspath{ {\main/_Images/} }

\begin{document}

\section{[Insert Section One Title]}

\sectiontitles{
    \sectiontitleselected{[ Insert Section One Title]}
    \sectiontitle{[Insert Section Two Title]}
    \sectiontitle{[Insert Section Three Title]}
    \sectiontitle{Lab - [Insert Chapter Title]}
}


%% [Insert Topic]

\subsection{[Insert Topic One Title]}

% figure - center a figure on the page.
\fig{_Workbook/nsbanner}

% codeblock - specify the language.
\begin{codeblock}{javascript}
// add.js

let x = 5
let y = 10
let s1 = 'for this is a string with \'embedded quotes\''
let s2 = 'for this is a string with "embedded quotes"'

console.log(x + 5) 
\end{codeblock}

\begin{codeblock}{bash}
$ node add.js
15
$
\end{codeblock}

\begin{displaynote}
The \$ character will be used as the command line prompt in all examples.
\end{displaynote}

% \begin{table}[H]
%     \resizebox{\tablewidth}{!}{
%         \hspace*{10pt}\begin{minipage}{\textwidth}
%             \def\arraystretch{1.5}
%             \rowcolors{2}{codeblock}{white}
%             \begin{tabularx}{\textwidth}{|l|X|}
%                 \hline
%                 \textbf{Header Column One} & \textbf{Header Column Two}\\
%                 \hline
%                 Row one, column one & Row one, column two\\
%                 \hline
%                 Row two, column one & Row two, column two\\
%                 \hline
%             \end{tabularx}
%         \end{minipage}
%     }
% \end{table}

\topictable{|l|X|}{
    \tableheader{Header Column One}{Header Column Two}
    \tablerow{Row one, column one}{Row one, column two}
    \tablerow{Row two, column one}{Row two, column two}
}

% The codeblock, table, or image is usually followed with a list of one to three bullet points. The command bulletlist, and
% the environment bullets, close up the space between the items. Using the environment is preferred, because you cannot nest
% an enumeration inside the bulletlist command.

\bulletlist{
    \item This is the first bullet point of the topic
    \item This is the second bullet point of the topic; use \emph{\\emph} to emphasise keywords in the topics (which should be the same as \emph{\\textit}), and \textbf{\\textbf} if bold is necessary
}

\begin{bullets}
    \item This is the first bullet point of the topic
    \item This is the second bullet point of the topic; use \emph{\\emph} to emphasise keywords in the topics (which should be the same as \emph{\\textit}), and \textbf{\\textbf} if bold is necessary
\end{bullets}

% The numbers environment closes up the space between the lines a bit over using the enumerate environment.

\begin{numbers}
    \item This is the first bullet point of the topic
    \item This is the second bullet point of the topic; use \emph{\\emph} to emphasise keywords in the topics (which should be the same as \emph{\\textit}), and \textbf{\\textbf} if bold is necessary
\end{numbers}

%% [Insert Second Topic..., follow topics with a checkpoint (in section-2) if there is only one section]

\subsection{[Insert Topic Two Title]}

\bulletlist{
    \item Each topic must have content
}


\end{document}
